\documentclass[utf8x, 12pt]{G7-32} 


% --------- -------- SETTINGS --------- --------

% --------- Настройки стиля ГОСТ 7-32 --------

% Гипертекстовое оглавление в PDF
\usepackage[
bookmarks=true, colorlinks=true, unicode=true,
urlcolor=black,linkcolor=black, anchorcolor=black,
citecolor=black, menucolor=black, filecolor=black,
]{hyperref}

\usepackage{graphicx}   % Пакет для включения рисунков
\DeclareGraphicsExtensions{.jpg,.pdf,.png}
\geometry{right=20mm}
\geometry{left=30mm}
\usepackage{enumerate}
\setcounter{tocdepth}{3} % Подробность оглавления


% --------- other settings --------
\usepackage{MnSymbol}
%\usepackage{simpsons}
% --------- -------- SETTINGS --------- --------



\begin{document}

\frontmatter 

% --------- -------- TITLE --------- --------

\begin{center} 

\large САНКТ-ПЕТЕРБУРГСИЙ ГОСУДАРСТВЕННЫЙ ПОЛИТЕХНИЧЕСКИЙ УНИВЕРСИТЕТ

\large Кафедра Компьютерных Систем и Программных Технологий \\[5.5cm] 

\huge ОТЧЕТ \\[0.6cm] % название работы, затем отступ 0,6см
\large по лабораторной работе №1\\
\large Тема: <<Система вёрстки \TeX и расширения \LaTeX>>\\
\large Дисциплина: <<Методы и средства защиты информации>>\\[3.7cm]

\end{center} 

\begin{flushright}
Выполнил: студент гр. 53501/2 \\
Федоров Е.М. \\[1.2cm]


Преподаватель \\
Вылегжанина К.Д.
\end{flushright}


\vfill 

\begin{center} 
\large Санкт-Петербург \\
2015
\end{center} 

\thispagestyle{empty}


% --------- -------- TITLE --------- --------

\thispagestyle{empty}
\setcounter{page}{0}
\tableofcontents
\clearpage
\mainmatter



\chapter{Опыт работы с \TeX}

До начала обучения по курсу защиты информации я никогда не пользовался ни редактором TexMaker, ни самим форматом \TeX . В основном я пользовался текстовым процессором MS Word, чьих возможностей по вёрстке текста для меня было более чем достатточно. \hfil\break
В некоторых случаях для верстки текста я пользовался Adobe Illustrator.

\chapter{Составление формул в \TeX}

Составление больших формул в \TeX является сложной задачей, однако при работе с небольшими формулами очень удобно их редактировать и переносить.\hfil\break
\begin{gather}
A_x = -Hy \\
A_y = A_z = 0
\end{gather}
\begin{eqnarray*}
& \int(F_i x_k - F_k x_i)\,dV = & \\
& \qquad=\oint(u_{il}x_k-u_{kl}x_i)\,df_l &
\end{eqnarray*}



\chapter{Положительные впечатления}

\TeX мощный инструмент вёрстки, имеющий большое количество положительных моментов:

\begin{enumerate}
	\item  Хорошая переносимость документов.
	\item Легкость в работе при использовании несложных формул.
	\item Легкость в создании и использовании собственных стилей форматирования.
	
\end{enumerate}



\chapter{Отрицательные впечатления}

Также существуют некоторое количество отрицательных моментов: 
\begin{enumerate}
	\item Для того чтобы пользоваться \TeX необходимо освоить сложную систему комманд.
	\item Большинство людей пользуются форматом doc, при необходимости редактирования Вашего файла другим человеком необходимо убедиться, что он в достаточной степени знаком с \TeX 
	\item Отсутствие проверки пунктуации в фокументе.
	

\end{enumerate}


\end{document}
